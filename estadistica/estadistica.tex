
% Default to the notebook output style

    


% Inherit from the specified cell style.




    
\documentclass[11pt]{article}

    \usepackage[spanish]{babel}
    
    \usepackage[T1]{fontenc}
    % Nicer default font (+ math font) than Computer Modern for most use cases
    \usepackage{mathpazo}

    % Basic figure setup, for now with no caption control since it's done
    % automatically by Pandoc (which extracts ![](path) syntax from Markdown).
    \usepackage{graphicx}
    % We will generate all images so they have a width \maxwidth. This means
    % that they will get their normal width if they fit onto the page, but
    % are scaled down if they would overflow the margins.
    \makeatletter
    \def\maxwidth{\ifdim\Gin@nat@width>\linewidth\linewidth
    \else\Gin@nat@width\fi}
    \makeatother
    \let\Oldincludegraphics\includegraphics
    % Set max figure width to be 80% of text width, for now hardcoded.
    \renewcommand{\includegraphics}[1]{\Oldincludegraphics[width=.8\maxwidth]{#1}}
    % Ensure that by default, figures have no caption (until we provide a
    % proper Figure object with a Caption API and a way to capture that
    % in the conversion process - todo).
    \usepackage{caption}
    \DeclareCaptionLabelFormat{nolabel}{}
    \captionsetup{labelformat=nolabel}

    \usepackage{adjustbox} % Used to constrain images to a maximum size 
    \usepackage{xcolor} % Allow colors to be defined
    \usepackage{enumerate} % Needed for markdown enumerations to work
    \usepackage{geometry} % Used to adjust the document margins
    \usepackage{amsmath} % Equations
    \usepackage{amssymb} % Equations
    \usepackage{textcomp} % defines textquotesingle
    % Hack from http://tex.stackexchange.com/a/47451/13684:
    \AtBeginDocument{%
        \def\PYZsq{\textquotesingle}% Upright quotes in Pygmentized code
    }
    \usepackage{upquote} % Upright quotes for verbatim code
    \usepackage{eurosym} % defines \euro
    \usepackage[mathletters]{ucs} % Extended unicode (utf-8) support
    \usepackage[utf8x]{inputenc} % Allow utf-8 characters in the tex document
    \usepackage{fancyvrb} % verbatim replacement that allows latex
    \usepackage{grffile} % extends the file name processing of package graphics 
                         % to support a larger range 
    % The hyperref package gives us a pdf with properly built
    % internal navigation ('pdf bookmarks' for the table of contents,
    % internal cross-reference links, web links for URLs, etc.)
    \usepackage{hyperref}
    \usepackage{longtable} % longtable support required by pandoc >1.10
    \usepackage{booktabs}  % table support for pandoc > 1.12.2
    \usepackage[inline]{enumitem} % IRkernel/repr support (it uses the enumerate* environment)
    \usepackage[normalem]{ulem} % ulem is needed to support strikethroughs (\sout)
                                % normalem makes italics be italics, not underlines
    \usepackage{mathrsfs}
    

    
    
    % Colors for the hyperref package
    \definecolor{urlcolor}{rgb}{0,.145,.698}
    \definecolor{linkcolor}{rgb}{.71,0.21,0.01}
    \definecolor{citecolor}{rgb}{.12,.54,.11}

    % ANSI colors
    \definecolor{ansi-black}{HTML}{3E424D}
    \definecolor{ansi-black-intense}{HTML}{282C36}
    \definecolor{ansi-red}{HTML}{E75C58}
    \definecolor{ansi-red-intense}{HTML}{B22B31}
    \definecolor{ansi-green}{HTML}{00A250}
    \definecolor{ansi-green-intense}{HTML}{007427}
    \definecolor{ansi-yellow}{HTML}{DDB62B}
    \definecolor{ansi-yellow-intense}{HTML}{B27D12}
    \definecolor{ansi-blue}{HTML}{208FFB}
    \definecolor{ansi-blue-intense}{HTML}{0065CA}
    \definecolor{ansi-magenta}{HTML}{D160C4}
    \definecolor{ansi-magenta-intense}{HTML}{A03196}
    \definecolor{ansi-cyan}{HTML}{60C6C8}
    \definecolor{ansi-cyan-intense}{HTML}{258F8F}
    \definecolor{ansi-white}{HTML}{C5C1B4}
    \definecolor{ansi-white-intense}{HTML}{A1A6B2}
    \definecolor{ansi-default-inverse-fg}{HTML}{FFFFFF}
    \definecolor{ansi-default-inverse-bg}{HTML}{000000}

    % commands and environments needed by pandoc snippets
    % extracted from the output of `pandoc -s`
    \providecommand{\tightlist}{%
      \setlength{\itemsep}{0pt}\setlength{\parskip}{0pt}}
    \DefineVerbatimEnvironment{Highlighting}{Verbatim}{commandchars=\\\{\}}
    % Add ',fontsize=\small' for more characters per line
    \newenvironment{Shaded}{}{}
    \newcommand{\KeywordTok}[1]{\textcolor[rgb]{0.00,0.44,0.13}{\textbf{{#1}}}}
    \newcommand{\DataTypeTok}[1]{\textcolor[rgb]{0.56,0.13,0.00}{{#1}}}
    \newcommand{\DecValTok}[1]{\textcolor[rgb]{0.25,0.63,0.44}{{#1}}}
    \newcommand{\BaseNTok}[1]{\textcolor[rgb]{0.25,0.63,0.44}{{#1}}}
    \newcommand{\FloatTok}[1]{\textcolor[rgb]{0.25,0.63,0.44}{{#1}}}
    \newcommand{\CharTok}[1]{\textcolor[rgb]{0.25,0.44,0.63}{{#1}}}
    \newcommand{\StringTok}[1]{\textcolor[rgb]{0.25,0.44,0.63}{{#1}}}
    \newcommand{\CommentTok}[1]{\textcolor[rgb]{0.38,0.63,0.69}{\textit{{#1}}}}
    \newcommand{\OtherTok}[1]{\textcolor[rgb]{0.00,0.44,0.13}{{#1}}}
    \newcommand{\AlertTok}[1]{\textcolor[rgb]{1.00,0.00,0.00}{\textbf{{#1}}}}
    \newcommand{\FunctionTok}[1]{\textcolor[rgb]{0.02,0.16,0.49}{{#1}}}
    \newcommand{\RegionMarkerTok}[1]{{#1}}
    \newcommand{\ErrorTok}[1]{\textcolor[rgb]{1.00,0.00,0.00}{\textbf{{#1}}}}
    \newcommand{\NormalTok}[1]{{#1}}
    
    % Additional commands for more recent versions of Pandoc
    \newcommand{\ConstantTok}[1]{\textcolor[rgb]{0.53,0.00,0.00}{{#1}}}
    \newcommand{\SpecialCharTok}[1]{\textcolor[rgb]{0.25,0.44,0.63}{{#1}}}
    \newcommand{\VerbatimStringTok}[1]{\textcolor[rgb]{0.25,0.44,0.63}{{#1}}}
    \newcommand{\SpecialStringTok}[1]{\textcolor[rgb]{0.73,0.40,0.53}{{#1}}}
    \newcommand{\ImportTok}[1]{{#1}}
    \newcommand{\DocumentationTok}[1]{\textcolor[rgb]{0.73,0.13,0.13}{\textit{{#1}}}}
    \newcommand{\AnnotationTok}[1]{\textcolor[rgb]{0.38,0.63,0.69}{\textbf{\textit{{#1}}}}}
    \newcommand{\CommentVarTok}[1]{\textcolor[rgb]{0.38,0.63,0.69}{\textbf{\textit{{#1}}}}}
    \newcommand{\VariableTok}[1]{\textcolor[rgb]{0.10,0.09,0.49}{{#1}}}
    \newcommand{\ControlFlowTok}[1]{\textcolor[rgb]{0.00,0.44,0.13}{\textbf{{#1}}}}
    \newcommand{\OperatorTok}[1]{\textcolor[rgb]{0.40,0.40,0.40}{{#1}}}
    \newcommand{\BuiltInTok}[1]{{#1}}
    \newcommand{\ExtensionTok}[1]{{#1}}
    \newcommand{\PreprocessorTok}[1]{\textcolor[rgb]{0.74,0.48,0.00}{{#1}}}
    \newcommand{\AttributeTok}[1]{\textcolor[rgb]{0.49,0.56,0.16}{{#1}}}
    \newcommand{\InformationTok}[1]{\textcolor[rgb]{0.38,0.63,0.69}{\textbf{\textit{{#1}}}}}
    \newcommand{\WarningTok}[1]{\textcolor[rgb]{0.38,0.63,0.69}{\textbf{\textit{{#1}}}}}
    
    
    % Define a nice break command that doesn't care if a line doesn't already
    % exist.
    \def\br{\hspace*{\fill} \\* }
    % Math Jax compatibility definitions
    \def\gt{>}
    \def\lt{<}
    \let\Oldtex\TeX
    \let\Oldlatex\LaTeX
    \renewcommand{\TeX}{\textrm{\Oldtex}}
    \renewcommand{\LaTeX}{\textrm{\Oldlatex}}
    % Document parameters
    % Document title
    \title{Ecocardiogramas para detectar riesgo de mortalidad}
    \author{Rafael Alejandro Castillo L\'opez}
    
    
    
    

    % Pygments definitions
    
\makeatletter
\def\PY@reset{\let\PY@it=\relax \let\PY@bf=\relax%
    \let\PY@ul=\relax \let\PY@tc=\relax%
    \let\PY@bc=\relax \let\PY@ff=\relax}
\def\PY@tok#1{\csname PY@tok@#1\endcsname}
\def\PY@toks#1+{\ifx\relax#1\empty\else%
    \PY@tok{#1}\expandafter\PY@toks\fi}
\def\PY@do#1{\PY@bc{\PY@tc{\PY@ul{%
    \PY@it{\PY@bf{\PY@ff{#1}}}}}}}
\def\PY#1#2{\PY@reset\PY@toks#1+\relax+\PY@do{#2}}

\expandafter\def\csname PY@tok@w\endcsname{\def\PY@tc##1{\textcolor[rgb]{0.73,0.73,0.73}{##1}}}
\expandafter\def\csname PY@tok@c\endcsname{\let\PY@it=\textit\def\PY@tc##1{\textcolor[rgb]{0.25,0.50,0.50}{##1}}}
\expandafter\def\csname PY@tok@cp\endcsname{\def\PY@tc##1{\textcolor[rgb]{0.74,0.48,0.00}{##1}}}
\expandafter\def\csname PY@tok@k\endcsname{\let\PY@bf=\textbf\def\PY@tc##1{\textcolor[rgb]{0.00,0.50,0.00}{##1}}}
\expandafter\def\csname PY@tok@kp\endcsname{\def\PY@tc##1{\textcolor[rgb]{0.00,0.50,0.00}{##1}}}
\expandafter\def\csname PY@tok@kt\endcsname{\def\PY@tc##1{\textcolor[rgb]{0.69,0.00,0.25}{##1}}}
\expandafter\def\csname PY@tok@o\endcsname{\def\PY@tc##1{\textcolor[rgb]{0.40,0.40,0.40}{##1}}}
\expandafter\def\csname PY@tok@ow\endcsname{\let\PY@bf=\textbf\def\PY@tc##1{\textcolor[rgb]{0.67,0.13,1.00}{##1}}}
\expandafter\def\csname PY@tok@nb\endcsname{\def\PY@tc##1{\textcolor[rgb]{0.00,0.50,0.00}{##1}}}
\expandafter\def\csname PY@tok@nf\endcsname{\def\PY@tc##1{\textcolor[rgb]{0.00,0.00,1.00}{##1}}}
\expandafter\def\csname PY@tok@nc\endcsname{\let\PY@bf=\textbf\def\PY@tc##1{\textcolor[rgb]{0.00,0.00,1.00}{##1}}}
\expandafter\def\csname PY@tok@nn\endcsname{\let\PY@bf=\textbf\def\PY@tc##1{\textcolor[rgb]{0.00,0.00,1.00}{##1}}}
\expandafter\def\csname PY@tok@ne\endcsname{\let\PY@bf=\textbf\def\PY@tc##1{\textcolor[rgb]{0.82,0.25,0.23}{##1}}}
\expandafter\def\csname PY@tok@nv\endcsname{\def\PY@tc##1{\textcolor[rgb]{0.10,0.09,0.49}{##1}}}
\expandafter\def\csname PY@tok@no\endcsname{\def\PY@tc##1{\textcolor[rgb]{0.53,0.00,0.00}{##1}}}
\expandafter\def\csname PY@tok@nl\endcsname{\def\PY@tc##1{\textcolor[rgb]{0.63,0.63,0.00}{##1}}}
\expandafter\def\csname PY@tok@ni\endcsname{\let\PY@bf=\textbf\def\PY@tc##1{\textcolor[rgb]{0.60,0.60,0.60}{##1}}}
\expandafter\def\csname PY@tok@na\endcsname{\def\PY@tc##1{\textcolor[rgb]{0.49,0.56,0.16}{##1}}}
\expandafter\def\csname PY@tok@nt\endcsname{\let\PY@bf=\textbf\def\PY@tc##1{\textcolor[rgb]{0.00,0.50,0.00}{##1}}}
\expandafter\def\csname PY@tok@nd\endcsname{\def\PY@tc##1{\textcolor[rgb]{0.67,0.13,1.00}{##1}}}
\expandafter\def\csname PY@tok@s\endcsname{\def\PY@tc##1{\textcolor[rgb]{0.73,0.13,0.13}{##1}}}
\expandafter\def\csname PY@tok@sd\endcsname{\let\PY@it=\textit\def\PY@tc##1{\textcolor[rgb]{0.73,0.13,0.13}{##1}}}
\expandafter\def\csname PY@tok@si\endcsname{\let\PY@bf=\textbf\def\PY@tc##1{\textcolor[rgb]{0.73,0.40,0.53}{##1}}}
\expandafter\def\csname PY@tok@se\endcsname{\let\PY@bf=\textbf\def\PY@tc##1{\textcolor[rgb]{0.73,0.40,0.13}{##1}}}
\expandafter\def\csname PY@tok@sr\endcsname{\def\PY@tc##1{\textcolor[rgb]{0.73,0.40,0.53}{##1}}}
\expandafter\def\csname PY@tok@ss\endcsname{\def\PY@tc##1{\textcolor[rgb]{0.10,0.09,0.49}{##1}}}
\expandafter\def\csname PY@tok@sx\endcsname{\def\PY@tc##1{\textcolor[rgb]{0.00,0.50,0.00}{##1}}}
\expandafter\def\csname PY@tok@m\endcsname{\def\PY@tc##1{\textcolor[rgb]{0.40,0.40,0.40}{##1}}}
\expandafter\def\csname PY@tok@gh\endcsname{\let\PY@bf=\textbf\def\PY@tc##1{\textcolor[rgb]{0.00,0.00,0.50}{##1}}}
\expandafter\def\csname PY@tok@gu\endcsname{\let\PY@bf=\textbf\def\PY@tc##1{\textcolor[rgb]{0.50,0.00,0.50}{##1}}}
\expandafter\def\csname PY@tok@gd\endcsname{\def\PY@tc##1{\textcolor[rgb]{0.63,0.00,0.00}{##1}}}
\expandafter\def\csname PY@tok@gi\endcsname{\def\PY@tc##1{\textcolor[rgb]{0.00,0.63,0.00}{##1}}}
\expandafter\def\csname PY@tok@gr\endcsname{\def\PY@tc##1{\textcolor[rgb]{1.00,0.00,0.00}{##1}}}
\expandafter\def\csname PY@tok@ge\endcsname{\let\PY@it=\textit}
\expandafter\def\csname PY@tok@gs\endcsname{\let\PY@bf=\textbf}
\expandafter\def\csname PY@tok@gp\endcsname{\let\PY@bf=\textbf\def\PY@tc##1{\textcolor[rgb]{0.00,0.00,0.50}{##1}}}
\expandafter\def\csname PY@tok@go\endcsname{\def\PY@tc##1{\textcolor[rgb]{0.53,0.53,0.53}{##1}}}
\expandafter\def\csname PY@tok@gt\endcsname{\def\PY@tc##1{\textcolor[rgb]{0.00,0.27,0.87}{##1}}}
\expandafter\def\csname PY@tok@err\endcsname{\def\PY@bc##1{\setlength{\fboxsep}{0pt}\fcolorbox[rgb]{1.00,0.00,0.00}{1,1,1}{\strut ##1}}}
\expandafter\def\csname PY@tok@kc\endcsname{\let\PY@bf=\textbf\def\PY@tc##1{\textcolor[rgb]{0.00,0.50,0.00}{##1}}}
\expandafter\def\csname PY@tok@kd\endcsname{\let\PY@bf=\textbf\def\PY@tc##1{\textcolor[rgb]{0.00,0.50,0.00}{##1}}}
\expandafter\def\csname PY@tok@kn\endcsname{\let\PY@bf=\textbf\def\PY@tc##1{\textcolor[rgb]{0.00,0.50,0.00}{##1}}}
\expandafter\def\csname PY@tok@kr\endcsname{\let\PY@bf=\textbf\def\PY@tc##1{\textcolor[rgb]{0.00,0.50,0.00}{##1}}}
\expandafter\def\csname PY@tok@bp\endcsname{\def\PY@tc##1{\textcolor[rgb]{0.00,0.50,0.00}{##1}}}
\expandafter\def\csname PY@tok@fm\endcsname{\def\PY@tc##1{\textcolor[rgb]{0.00,0.00,1.00}{##1}}}
\expandafter\def\csname PY@tok@vc\endcsname{\def\PY@tc##1{\textcolor[rgb]{0.10,0.09,0.49}{##1}}}
\expandafter\def\csname PY@tok@vg\endcsname{\def\PY@tc##1{\textcolor[rgb]{0.10,0.09,0.49}{##1}}}
\expandafter\def\csname PY@tok@vi\endcsname{\def\PY@tc##1{\textcolor[rgb]{0.10,0.09,0.49}{##1}}}
\expandafter\def\csname PY@tok@vm\endcsname{\def\PY@tc##1{\textcolor[rgb]{0.10,0.09,0.49}{##1}}}
\expandafter\def\csname PY@tok@sa\endcsname{\def\PY@tc##1{\textcolor[rgb]{0.73,0.13,0.13}{##1}}}
\expandafter\def\csname PY@tok@sb\endcsname{\def\PY@tc##1{\textcolor[rgb]{0.73,0.13,0.13}{##1}}}
\expandafter\def\csname PY@tok@sc\endcsname{\def\PY@tc##1{\textcolor[rgb]{0.73,0.13,0.13}{##1}}}
\expandafter\def\csname PY@tok@dl\endcsname{\def\PY@tc##1{\textcolor[rgb]{0.73,0.13,0.13}{##1}}}
\expandafter\def\csname PY@tok@s2\endcsname{\def\PY@tc##1{\textcolor[rgb]{0.73,0.13,0.13}{##1}}}
\expandafter\def\csname PY@tok@sh\endcsname{\def\PY@tc##1{\textcolor[rgb]{0.73,0.13,0.13}{##1}}}
\expandafter\def\csname PY@tok@s1\endcsname{\def\PY@tc##1{\textcolor[rgb]{0.73,0.13,0.13}{##1}}}
\expandafter\def\csname PY@tok@mb\endcsname{\def\PY@tc##1{\textcolor[rgb]{0.40,0.40,0.40}{##1}}}
\expandafter\def\csname PY@tok@mf\endcsname{\def\PY@tc##1{\textcolor[rgb]{0.40,0.40,0.40}{##1}}}
\expandafter\def\csname PY@tok@mh\endcsname{\def\PY@tc##1{\textcolor[rgb]{0.40,0.40,0.40}{##1}}}
\expandafter\def\csname PY@tok@mi\endcsname{\def\PY@tc##1{\textcolor[rgb]{0.40,0.40,0.40}{##1}}}
\expandafter\def\csname PY@tok@il\endcsname{\def\PY@tc##1{\textcolor[rgb]{0.40,0.40,0.40}{##1}}}
\expandafter\def\csname PY@tok@mo\endcsname{\def\PY@tc##1{\textcolor[rgb]{0.40,0.40,0.40}{##1}}}
\expandafter\def\csname PY@tok@ch\endcsname{\let\PY@it=\textit\def\PY@tc##1{\textcolor[rgb]{0.25,0.50,0.50}{##1}}}
\expandafter\def\csname PY@tok@cm\endcsname{\let\PY@it=\textit\def\PY@tc##1{\textcolor[rgb]{0.25,0.50,0.50}{##1}}}
\expandafter\def\csname PY@tok@cpf\endcsname{\let\PY@it=\textit\def\PY@tc##1{\textcolor[rgb]{0.25,0.50,0.50}{##1}}}
\expandafter\def\csname PY@tok@c1\endcsname{\let\PY@it=\textit\def\PY@tc##1{\textcolor[rgb]{0.25,0.50,0.50}{##1}}}
\expandafter\def\csname PY@tok@cs\endcsname{\let\PY@it=\textit\def\PY@tc##1{\textcolor[rgb]{0.25,0.50,0.50}{##1}}}

\def\PYZbs{\char`\\}
\def\PYZus{\char`\_}
\def\PYZob{\char`\{}
\def\PYZcb{\char`\}}
\def\PYZca{\char`\^}
\def\PYZam{\char`\&}
\def\PYZlt{\char`\<}
\def\PYZgt{\char`\>}
\def\PYZsh{\char`\#}
\def\PYZpc{\char`\%}
\def\PYZdl{\char`\$}
\def\PYZhy{\char`\-}
\def\PYZsq{\char`\'}
\def\PYZdq{\char`\"}
\def\PYZti{\char`\~}
% for compatibility with earlier versions
\def\PYZat{@}
\def\PYZlb{[}
\def\PYZrb{]}
\makeatother


    % Exact colors from NB
    \definecolor{incolor}{rgb}{0.0, 0.0, 0.5}
    \definecolor{outcolor}{rgb}{0.545, 0.0, 0.0}



    
    % Prevent overflowing lines due to hard-to-break entities
    \sloppy 
    % Setup hyperref package
    \hypersetup{
      breaklinks=true,  % so long urls are correctly broken across lines
      colorlinks=true,
      urlcolor=urlcolor,
      linkcolor=linkcolor,
      citecolor=citecolor,
      }
    % Slightly bigger margins than the latex defaults
    
    \geometry{verbose,tmargin=1in,bmargin=1in,lmargin=1in,rmargin=1in}
    
    

    \begin{document}
    

\maketitle

\begin{abstract}
Se analizó un conjunto de datos disponible publicamente sobre el uso de ecocardiogramas para la detección de riesgo de mortalidad dentro de pacientes que ya han sufrido ataques cardiacos. Se realizan varias pruebas estadisticas para detectar que los conjuntos de datos son normales y que una prueba de hipotesis nos confirma una relación entre las variables detectadas por un ecocardiograma y si el paciente fallece en el periodo de un año despues de su primer incidente.
\end{abstract}
    
    \hypertarget{introducciuxf3n}{%
\section{Introducción}\label{introducciuxf3n}}

Un ecocardiograma es un sonograma del corazon. La echocardiografia se
usa rutinariamente para el diagnosi y manejo de pacientes con
enfermedades cardiacas conocidas o sospechadas. Es una de las pruebas
mas usadas en la cardiologia. Puede proveer una gran cantidad de
información, incluyendo el tamaño y forma del corazon, capacidad de
bombeo, y la ubicación y severidad del daño al tejido. Un ecocardiograma
tambien puede darle a un medico otras estimaciones del funcionamiento
cardiaco, como el gasto cardiaco, la fraccion de eyección, y la función
diastolica.

Tomaremos en cuenta un estudio realizado por \emph{Kan et al}, que
consiste en una muestra de ecocardiogramas de 132 pacientes que han
tenido ataques al corazón. Se incluyen varias observaciones obtenidas
del ecocardiograma, asi como el tiempo de supervivencia del paciente.
Las dos categorias que consideraremos son los pacientes que
sobrevivieron o no al primer año despues de sufrir un ataque cardiaco.

    \hypertarget{metodologia}{%
\section{Herramientas}\label{Herramientas}}

El analisis lo realicé en el lenguaje de programación Python. Para
cargar y manipular los datos usé Pandas, la librería mas popular para
propositos de manipulación de datos. Para generar visualizaciónes se usó
MatPlotLib y Seaborn, dos de los paquetes mas populares para este
proposito.

Otras herramientas que usamos son numpy para manipulación númerica y
scipy, que contiene subrutinas para pruebas estadisticas. Del paquete
SciKit learn se usaron las subrutinas usadas para ajustar los modelos de
regresión.

Tambien notable es que este reporte fue producido directamente a partir de una libreta de Jupyter. Todo el codigo y texto necesario para crearlo se encuentran en el archivo de jupyter anexa.

    \begin{Verbatim}[commandchars=\\\{\}]
{\color{incolor}In [{\color{incolor}1}]:} \PY{k+kn}{import} \PY{n+nn}{pandas} \PY{k}{as} \PY{n+nn}{pd}
        \PY{k+kn}{import} \PY{n+nn}{holoviews} \PY{k}{as} \PY{n+nn}{hv}
        \PY{k+kn}{import} \PY{n+nn}{scipy}\PY{n+nn}{.}\PY{n+nn}{stats}
        \PY{k+kn}{import} \PY{n+nn}{numpy} \PY{k}{as} \PY{n+nn}{np}
        \PY{k+kn}{import} \PY{n+nn}{matplotlib}\PY{n+nn}{.}\PY{n+nn}{pyplot} \PY{k}{as} \PY{n+nn}{plt}
        \PY{k+kn}{import} \PY{n+nn}{seaborn} \PY{k}{as} \PY{n+nn}{sns}
        
        \PY{n}{pd}\PY{o}{.}\PY{n}{set\PYZus{}option}\PY{p}{(}\PY{l+s+s1}{\PYZsq{}}\PY{l+s+s1}{display.notebook\PYZus{}repr\PYZus{}html}\PY{l+s+s1}{\PYZsq{}}\PY{p}{,} \PY{k+kc}{True}\PY{p}{)}
        
        \PY{k}{def} \PY{n+nf}{\PYZus{}repr\PYZus{}latex\PYZus{}}\PY{p}{(}\PY{n+nb+bp}{self}\PY{p}{)}\PY{p}{:}
            \PY{k}{return} \PY{n+nb+bp}{self}\PY{o}{.}\PY{n}{to\PYZus{}latex}\PY{p}{(}\PY{p}{)}
        
        \PY{n}{pd}\PY{o}{.}\PY{n}{DataFrame}\PY{o}{.}\PY{n}{\PYZus{}repr\PYZus{}latex\PYZus{}} \PY{o}{=} \PY{n}{\PYZus{}repr\PYZus{}latex\PYZus{}}
\end{Verbatim}

    Los datos fueron obtenidos del repositorio de datos de la University of
California Irvine. El conjunto de datos disponible en linea no es el
original, que incluye 350 pacientes, sino una versión alternativa
recolectada y donada a UCI por Salzberg.

Las variables incluidas son: - \texttt{survival}: indica la cantidad de
meses que el paciente sobrevivió/ha sobrevivido. - \texttt{still\_alive}
indica si el paciente seguia vivo al momento de que se recolectaron los
datos. - \texttt{age\_at\_heart\_attack} indica la edad al que el
paciente tuvo el paro cardiaco. - \texttt{pericardial\_effusion} es
binario e indica si el paciente tenia fluido en exceso alrededor del
corazon. - \texttt{fractional\_shortening} es el acortamiento en el
diametro de la ventricula izquierda. Valores mas pequeños son mas
anormales. - \texttt{epss} separacion septal del punto e. -
\texttt{lvdd} medida del tamaño del corazón. -
\texttt{wall\_motion\_index} mide el movimiento de los segmentos del
ventriculo izquierdo. - \texttt{alive\_at\_one} indica si el paciente
sobrevivió un año despues de su paro cardiaco.

Para mantener el trabajo enfocado, nos concentraremos en tres variables:
edad, acortamiento de la ventricula y el indice de movimiento.
Buscaremos ver la relación que existe entre estas variables y el indice
de supervivencia de los pacientes en el primer año:

\[
\begin{split}
A_1\colon & \text{ Edad de los pacientes que sobrevivieron al año} \\
A_2\colon & \text{ Edad de los pacientes que no sobrevivieron al año} \\
B_1\colon & \text{ Movimiento ventriculares de los pacientes que sobrevivieron al año} \\
B_1\colon & \text{ Movimiento ventriculares de los pacientes que sobrevivieron al año} \\
C_1\colon & \text{ Acortamiento ventricular para pacientes sobrevivientes al año} \\
C_1\colon & \text{ Acortamiento ventricular para pacientes no sobrevivientes al año} \\
\end{split}
\]

    \hypertarget{analisis}{%
\section{Analisis}\label{analisis}}

\hypertarget{analisis-inicial}{%
\subsection{Analisis inicial}\label{analisis-inicial}}

Primero que nada veremos una muestra de los datos para hacernos una idea
de su forma. Cargamos los datos, nos deshacemos de las variables que no
trataremos, y veremos 5 renglones muestreados aleagoreamente

    \begin{Verbatim}[commandchars=\\\{\}]
{\color{incolor}In [{\color{incolor}2}]:} \PY{n}{df} \PY{o}{=} \PY{n}{pd}\PY{o}{.}\PY{n}{read\PYZus{}csv}\PY{p}{(}
            \PY{l+s+s1}{\PYZsq{}}\PY{l+s+s1}{fixed\PYZus{}echocardiogram.csv}\PY{l+s+s1}{\PYZsq{}}\PY{p}{,} \PY{n}{encoding} \PY{o}{=} \PY{l+s+s2}{\PYZdq{}}\PY{l+s+s2}{ISO\PYZhy{}8859\PYZhy{}1}\PY{l+s+s2}{\PYZdq{}}\PY{p}{,}
            \PY{n}{error\PYZus{}bad\PYZus{}lines}\PY{o}{=}\PY{k+kc}{False}\PY{p}{,} \PY{n}{index\PYZus{}col}\PY{o}{=}\PY{l+m+mi}{0}\PY{p}{)}
        \PY{n}{df} \PY{o}{=} \PY{n}{df}\PY{o}{.}\PY{n}{replace}\PY{p}{(}\PY{l+s+s1}{\PYZsq{}}\PY{l+s+s1}{?}\PY{l+s+s1}{\PYZsq{}}\PY{p}{,} \PY{k+kc}{None}\PY{p}{)}\PY{o}{.}\PY{n}{replace}\PY{p}{(}\PY{l+s+s1}{\PYZsq{}}\PY{l+s+s1}{name}\PY{l+s+s1}{\PYZsq{}}\PY{p}{,} \PY{k+kc}{None}\PY{p}{)}\PY{o}{.}\PY{n}{astype}\PY{p}{(}\PY{n+nb}{float}\PY{p}{)}
        \PY{n}{df}\PY{p}{[}\PY{l+s+s1}{\PYZsq{}}\PY{l+s+s1}{alive\PYZus{}at\PYZus{}one}\PY{l+s+s1}{\PYZsq{}}\PY{p}{]}\PY{p}{[}\PY{n}{df}\PY{p}{[}\PY{l+s+s1}{\PYZsq{}}\PY{l+s+s1}{alive\PYZus{}at\PYZus{}one}\PY{l+s+s1}{\PYZsq{}}\PY{p}{]} \PY{o}{==} \PY{l+m+mf}{0.0}\PY{p}{]} \PY{o}{=} \PY{l+s+s1}{\PYZsq{}}\PY{l+s+s1}{no}\PY{l+s+s1}{\PYZsq{}}
        \PY{n}{df}\PY{p}{[}\PY{l+s+s1}{\PYZsq{}}\PY{l+s+s1}{alive\PYZus{}at\PYZus{}one}\PY{l+s+s1}{\PYZsq{}}\PY{p}{]}\PY{p}{[}\PY{n}{df}\PY{p}{[}\PY{l+s+s1}{\PYZsq{}}\PY{l+s+s1}{alive\PYZus{}at\PYZus{}one}\PY{l+s+s1}{\PYZsq{}}\PY{p}{]} \PY{o}{==} \PY{l+m+mf}{1.0}\PY{p}{]} \PY{o}{=} \PY{l+s+s1}{\PYZsq{}}\PY{l+s+s1}{si}\PY{l+s+s1}{\PYZsq{}}
        
        \PY{n}{df} \PY{o}{=} \PY{n}{df}\PY{p}{[}\PY{n}{df}\PY{p}{[}\PY{l+s+s1}{\PYZsq{}}\PY{l+s+s1}{alive\PYZus{}at\PYZus{}one}\PY{l+s+s1}{\PYZsq{}}\PY{p}{]} \PY{o}{!=} \PY{l+m+mi}{2}\PY{p}{]}
        \PY{n}{df\PYZus{}vars} \PY{o}{=} \PY{n}{df}\PY{p}{[}\PY{p}{[}\PY{l+s+s1}{\PYZsq{}}\PY{l+s+s1}{age\PYZus{}at\PYZus{}heart\PYZus{}attack}\PY{l+s+s1}{\PYZsq{}}\PY{p}{,} \PY{l+s+s1}{\PYZsq{}}\PY{l+s+s1}{fractional\PYZus{}shortening}\PY{l+s+s1}{\PYZsq{}}\PY{p}{,}
                      \PY{l+s+s1}{\PYZsq{}}\PY{l+s+s1}{wall\PYZus{}motion\PYZus{}score}\PY{l+s+s1}{\PYZsq{}}\PY{p}{,} \PY{l+s+s1}{\PYZsq{}}\PY{l+s+s1}{alive\PYZus{}at\PYZus{}one}\PY{l+s+s1}{\PYZsq{}}\PY{p}{]}\PY{p}{]}
        \PY{n}{df\PYZus{}vars}\PY{o}{.}\PY{n}{sample}\PY{p}{(}\PY{l+m+mi}{5}\PY{p}{)}
\end{Verbatim}
\texttt{\color{outcolor}Out[{\color{outcolor}2}]:}
    
    \begin{tabular}{lrrrl}
\toprule
{} &  age\_at\_heart\_attack &  fractional\_shortening &  wall\_motion\_score & alive\_at\_one \\
\midrule
101 &            58.129698 &               0.374885 &           8.412793 &           si \\
85  &            63.370397 &               0.181984 &          11.036609 &           si \\
85  &            56.795282 &               0.219077 &           5.229008 &           no \\
44  &            64.395425 &               0.290655 &          14.587979 &           si \\
20  &            65.437999 &              -0.018644 &          15.248450 &           no \\
\bottomrule
\end{tabular}


    

    Primero observamos algunos datos utiles. La cantidad de pacientes que
sobrevivieron al primer año:

    \begin{Verbatim}[commandchars=\\\{\}]
{\color{incolor}In [{\color{incolor}3}]:} \PY{n}{df}\PY{p}{[}\PY{l+s+s1}{\PYZsq{}}\PY{l+s+s1}{alive\PYZus{}at\PYZus{}one}\PY{l+s+s1}{\PYZsq{}}\PY{p}{]}\PY{o}{.}\PY{n}{value\PYZus{}counts}\PY{p}{(}\PY{p}{)}
\end{Verbatim}

\begin{Verbatim}[commandchars=\\\{\}]
{\color{outcolor}Out[{\color{outcolor}3}]:} no    184
        si    166
        Name: alive\_at\_one, dtype: int64
\end{Verbatim}
            
    Ahora vemos nuestros valores medios y desviaciones de muestra para
nuestras las tres variables que escogimos:

    \begin{Verbatim}[commandchars=\\\{\}]
{\color{incolor}In [{\color{incolor}4}]:} \PY{n}{df\PYZus{}vars} \PY{o}{=} \PY{n}{df}\PY{p}{[}\PY{p}{[}\PY{l+s+s1}{\PYZsq{}}\PY{l+s+s1}{age\PYZus{}at\PYZus{}heart\PYZus{}attack}\PY{l+s+s1}{\PYZsq{}}\PY{p}{,} \PY{l+s+s1}{\PYZsq{}}\PY{l+s+s1}{fractional\PYZus{}shortening}\PY{l+s+s1}{\PYZsq{}}\PY{p}{,}
                      \PY{l+s+s1}{\PYZsq{}}\PY{l+s+s1}{wall\PYZus{}motion\PYZus{}score}\PY{l+s+s1}{\PYZsq{}}\PY{p}{,} \PY{l+s+s1}{\PYZsq{}}\PY{l+s+s1}{alive\PYZus{}at\PYZus{}one}\PY{l+s+s1}{\PYZsq{}}\PY{p}{]}\PY{p}{]}
        \PY{n}{df\PYZus{}vars}\PY{o}{.}\PY{n}{columns} \PY{o}{=} \PY{p}{[}\PY{l+s+s1}{\PYZsq{}}\PY{l+s+s1}{Edad}\PY{l+s+s1}{\PYZsq{}}\PY{p}{,} \PY{l+s+s1}{\PYZsq{}}\PY{l+s+s1}{Acortamiento fraccionario}\PY{l+s+s1}{\PYZsq{}}\PY{p}{,}
                           \PY{l+s+s1}{\PYZsq{}}\PY{l+s+s1}{Movimiento ventricular}\PY{l+s+s1}{\PYZsq{}}\PY{p}{,} \PY{l+s+s1}{\PYZsq{}}\PY{l+s+s1}{Vivo al año}\PY{l+s+s1}{\PYZsq{}}\PY{p}{]}
        
        
        \PY{n}{df\PYZus{}vars}\PY{o}{.}\PY{n}{groupby}\PY{p}{(}\PY{l+s+s1}{\PYZsq{}}\PY{l+s+s1}{Vivo al año}\PY{l+s+s1}{\PYZsq{}}\PY{p}{)}\PY{o}{.}\PY{n}{agg}\PY{p}{(}\PY{p}{[}\PY{l+s+s1}{\PYZsq{}}\PY{l+s+s1}{mean}\PY{l+s+s1}{\PYZsq{}}\PY{p}{,} \PY{l+s+s1}{\PYZsq{}}\PY{l+s+s1}{std}\PY{l+s+s1}{\PYZsq{}}\PY{p}{]}\PY{p}{)}
\end{Verbatim}
\texttt{\color{outcolor}Out[{\color{outcolor}4}]:}
    
    \begin{tabular}{lrrrrrr}
\toprule
{} & \multicolumn{2}{l}{Edad} & \multicolumn{2}{l}{Acortamiento fraccionario} & \multicolumn{2}{l}{Movimiento ventricular} \\
{} &       mean &       std &                      mean &       std &                   mean &       std \\
Vivo al año &            &           &                           &           &                        &           \\
\midrule
no          &  61.858149 &  7.940153 &                  0.238587 &  0.129248 &              13.569254 &  3.674553 \\
si          &  62.755676 &  8.284627 &                  0.197249 &  0.089193 &              15.071443 &  6.818312 \\
\bottomrule
\end{tabular}


    

    La primera observación que podemos hacer es que los valores de edad no
parecen ser muy diferentes, pero la historia es distinta para el
acortamiento fraccional y el movimiento, que parecen ser distintos en
ambas categorias. Podemos recurrir a un diagrama de cajas para
visualizar como son distintos.

    \begin{Verbatim}[commandchars=\\\{\}]
{\color{incolor}In [{\color{incolor}5}]:} \PY{n}{f}\PY{p}{,} \PY{n}{axes} \PY{o}{=} \PY{n}{plt}\PY{o}{.}\PY{n}{subplots}\PY{p}{(}\PY{l+m+mi}{1}\PY{p}{,} \PY{l+m+mi}{3}\PY{p}{,} \PY{n}{figsize}\PY{o}{=}\PY{p}{(}\PY{l+m+mi}{18}\PY{p}{,} \PY{l+m+mi}{5}\PY{p}{)}\PY{p}{)}
        \PY{n}{age\PYZus{}bp} \PY{o}{=} \PY{n}{sns}\PY{o}{.}\PY{n}{boxplot}\PY{p}{(}\PY{n}{x}\PY{o}{=}\PY{l+s+s2}{\PYZdq{}}\PY{l+s+s2}{Edad}\PY{l+s+s2}{\PYZdq{}}\PY{p}{,} \PY{n}{y}\PY{o}{=}\PY{l+s+s2}{\PYZdq{}}\PY{l+s+s2}{Vivo al año}\PY{l+s+s2}{\PYZdq{}}\PY{p}{,} \PY{n}{data}\PY{o}{=}\PY{n}{df\PYZus{}vars}\PY{p}{,}
                    \PY{n}{whis}\PY{o}{=}\PY{l+s+s2}{\PYZdq{}}\PY{l+s+s2}{range}\PY{l+s+s2}{\PYZdq{}}\PY{p}{,} \PY{n}{palette}\PY{o}{=}\PY{l+s+s2}{\PYZdq{}}\PY{l+s+s2}{vlag}\PY{l+s+s2}{\PYZdq{}}\PY{p}{,} \PY{n}{ax}\PY{o}{=}\PY{n}{axes}\PY{p}{[}\PY{l+m+mi}{0}\PY{p}{]}\PY{p}{)}
        \PY{n}{m\PYZus{}bp} \PY{o}{=} \PY{n}{sns}\PY{o}{.}\PY{n}{boxplot}\PY{p}{(}\PY{n}{x}\PY{o}{=}\PY{l+s+s2}{\PYZdq{}}\PY{l+s+s2}{Movimiento ventricular}\PY{l+s+s2}{\PYZdq{}}\PY{p}{,} \PY{n}{y}\PY{o}{=}\PY{l+s+s2}{\PYZdq{}}\PY{l+s+s2}{Vivo al año}\PY{l+s+s2}{\PYZdq{}}\PY{p}{,} \PY{n}{data}\PY{o}{=}\PY{n}{df\PYZus{}vars}\PY{p}{,}
                    \PY{n}{whis}\PY{o}{=}\PY{l+s+s2}{\PYZdq{}}\PY{l+s+s2}{range}\PY{l+s+s2}{\PYZdq{}}\PY{p}{,} \PY{n}{palette}\PY{o}{=}\PY{l+s+s2}{\PYZdq{}}\PY{l+s+s2}{vlag}\PY{l+s+s2}{\PYZdq{}}\PY{p}{,} \PY{n}{ax}\PY{o}{=}\PY{n}{axes}\PY{p}{[}\PY{l+m+mi}{1}\PY{p}{]}\PY{p}{)}
        \PY{n}{fs\PYZus{}bp} \PY{o}{=} \PY{n}{sns}\PY{o}{.}\PY{n}{boxplot}\PY{p}{(}\PY{n}{x}\PY{o}{=}\PY{l+s+s2}{\PYZdq{}}\PY{l+s+s2}{Acortamiento fraccionario}\PY{l+s+s2}{\PYZdq{}}\PY{p}{,} \PY{n}{y}\PY{o}{=}\PY{l+s+s2}{\PYZdq{}}\PY{l+s+s2}{Vivo al año}\PY{l+s+s2}{\PYZdq{}}\PY{p}{,} \PY{n}{data}\PY{o}{=}\PY{n}{df\PYZus{}vars}\PY{p}{,}
                    \PY{n}{whis}\PY{o}{=}\PY{l+s+s2}{\PYZdq{}}\PY{l+s+s2}{range}\PY{l+s+s2}{\PYZdq{}}\PY{p}{,} \PY{n}{palette}\PY{o}{=}\PY{l+s+s2}{\PYZdq{}}\PY{l+s+s2}{vlag}\PY{l+s+s2}{\PYZdq{}}\PY{p}{,} \PY{n}{ax}\PY{o}{=}\PY{n}{axes}\PY{p}{[}\PY{l+m+mi}{2}\PY{p}{]}\PY{p}{)}
        
        \PY{n}{age\PYZus{}bp}\PY{o}{.}\PY{n}{set\PYZus{}title}\PY{p}{(}\PY{l+s+s2}{\PYZdq{}}\PY{l+s+s2}{Distribución de edades}\PY{l+s+s2}{\PYZdq{}}\PY{p}{)}
        \PY{n}{m\PYZus{}bp}\PY{o}{.}\PY{n}{set\PYZus{}title}\PY{p}{(}\PY{l+s+s2}{\PYZdq{}}\PY{l+s+s2}{Distribución de movimientos ventriculares}\PY{l+s+s2}{\PYZdq{}}\PY{p}{)}
        \PY{n}{fs\PYZus{}bp}\PY{o}{.}\PY{n}{set\PYZus{}title}\PY{p}{(}\PY{l+s+s2}{\PYZdq{}}\PY{l+s+s2}{Distribución de A.F.}\PY{l+s+s2}{\PYZdq{}}\PY{p}{)}
        
        \PY{n+nb}{print}\PY{p}{(}\PY{p}{)}
\end{Verbatim}

    \begin{Verbatim}[commandchars=\\\{\}]


    \end{Verbatim}

    \begin{center}
    \adjustimage{max size={0.9\linewidth}{0.9\paperheight}}{output_11_1.png}
    \end{center}
    { \hspace*{\fill} \\}
    
    Podemos ver que en el caso de la edad las diferencias de las
distribuciones es muy ligera. Los otros dos se muestran mas prometedores
como indicadores de la supervivencia media de los pacientes.

\hypertarget{normalidad}{%
\subsection{Normalidad}\label{normalidad}}

Para saber que pruebas podemos usar, necesitamos probar la normalidad de
los datos. Para una visualizacion rapida, veremos histogramas para las
variables.

Histogramas para pacientes que sobrevivieron:

    \begin{Verbatim}[commandchars=\\\{\}]
{\color{incolor}In [{\color{incolor}6}]:} \PY{n}{df\PYZus{}alive} \PY{o}{=} \PY{n}{df\PYZus{}vars}\PY{p}{[}\PY{n}{df\PYZus{}vars}\PY{p}{[}\PY{l+s+s1}{\PYZsq{}}\PY{l+s+s1}{Vivo al año}\PY{l+s+s1}{\PYZsq{}}\PY{p}{]} \PY{o}{==} \PY{l+s+s1}{\PYZsq{}}\PY{l+s+s1}{si}\PY{l+s+s1}{\PYZsq{}}\PY{p}{]}
        \PY{n}{df\PYZus{}deceased} \PY{o}{=} \PY{n}{df\PYZus{}vars}\PY{p}{[}\PY{n}{df\PYZus{}vars}\PY{p}{[}\PY{l+s+s1}{\PYZsq{}}\PY{l+s+s1}{Vivo al año}\PY{l+s+s1}{\PYZsq{}}\PY{p}{]} \PY{o}{==} \PY{l+s+s1}{\PYZsq{}}\PY{l+s+s1}{no}\PY{l+s+s1}{\PYZsq{}}\PY{p}{]}
        
        \PY{n}{f}\PY{p}{,} \PY{n}{axes} \PY{o}{=} \PY{n}{plt}\PY{o}{.}\PY{n}{subplots}\PY{p}{(}\PY{l+m+mi}{2}\PY{p}{,} \PY{l+m+mi}{3}\PY{p}{,} \PY{n}{figsize}\PY{o}{=}\PY{p}{(}\PY{l+m+mi}{18}\PY{p}{,} \PY{l+m+mi}{10}\PY{p}{)}\PY{p}{)}
        
        \PY{n}{age\PYZus{}hist} \PY{o}{=} \PY{n}{sns}\PY{o}{.}\PY{n}{distplot}\PY{p}{(}\PY{n}{df\PYZus{}alive}\PY{p}{[}\PY{l+s+s1}{\PYZsq{}}\PY{l+s+s1}{Edad}\PY{l+s+s1}{\PYZsq{}}\PY{p}{]}\PY{p}{,} \PY{n}{ax}\PY{o}{=}\PY{n}{axes}\PY{p}{[}\PY{l+m+mi}{0}\PY{p}{]}\PY{p}{[}\PY{l+m+mi}{0}\PY{p}{]}\PY{p}{)}
        \PY{n}{wm\PYZus{}hist} \PY{o}{=} \PY{n}{sns}\PY{o}{.}\PY{n}{distplot}\PY{p}{(}\PY{n}{df\PYZus{}alive}\PY{p}{[}\PY{l+s+s1}{\PYZsq{}}\PY{l+s+s1}{Movimiento ventricular}\PY{l+s+s1}{\PYZsq{}}\PY{p}{]}\PY{p}{,} \PY{n}{ax}\PY{o}{=}\PY{n}{axes}\PY{p}{[}\PY{l+m+mi}{0}\PY{p}{]}\PY{p}{[}\PY{l+m+mi}{1}\PY{p}{]}\PY{p}{)}
        \PY{n}{fs\PYZus{}hist} \PY{o}{=} \PY{n}{sns}\PY{o}{.}\PY{n}{distplot}\PY{p}{(}\PY{n}{df\PYZus{}alive}\PY{p}{[}\PY{l+s+s1}{\PYZsq{}}\PY{l+s+s1}{Acortamiento fraccionario}\PY{l+s+s1}{\PYZsq{}}\PY{p}{]}\PY{p}{,} \PY{n}{ax}\PY{o}{=}\PY{n}{axes}\PY{p}{[}\PY{l+m+mi}{0}\PY{p}{]}\PY{p}{[}\PY{l+m+mi}{2}\PY{p}{]}\PY{p}{)}
        
        \PY{n}{age\PYZus{}hist}\PY{o}{.}\PY{n}{set\PYZus{}title}\PY{p}{(}\PY{l+s+s1}{\PYZsq{}}\PY{l+s+s1}{Edades de sobrevivientes al año}\PY{l+s+s1}{\PYZsq{}}\PY{p}{)}
        \PY{n}{wm\PYZus{}hist}\PY{o}{.}\PY{n}{set\PYZus{}title}\PY{p}{(}\PY{l+s+s1}{\PYZsq{}}\PY{l+s+s1}{Movimientos de sobrevivientes al año}\PY{l+s+s1}{\PYZsq{}}\PY{p}{)}
        \PY{n}{fs\PYZus{}hist}\PY{o}{.}\PY{n}{set\PYZus{}title}\PY{p}{(}\PY{l+s+s1}{\PYZsq{}}\PY{l+s+s1}{A.F. de sobrevivientes al año}\PY{l+s+s1}{\PYZsq{}}\PY{p}{)}
        
        \PY{n}{age\PYZus{}hist} \PY{o}{=} \PY{n}{sns}\PY{o}{.}\PY{n}{distplot}\PY{p}{(}\PY{n}{df\PYZus{}deceased}\PY{p}{[}\PY{l+s+s1}{\PYZsq{}}\PY{l+s+s1}{Edad}\PY{l+s+s1}{\PYZsq{}}\PY{p}{]}\PY{p}{,} \PY{n}{ax}\PY{o}{=}\PY{n}{axes}\PY{p}{[}\PY{l+m+mi}{1}\PY{p}{]}\PY{p}{[}\PY{l+m+mi}{0}\PY{p}{]}\PY{p}{)}
        \PY{n}{wm\PYZus{}hist} \PY{o}{=} \PY{n}{sns}\PY{o}{.}\PY{n}{distplot}\PY{p}{(}\PY{n}{df\PYZus{}deceased}\PY{p}{[}\PY{l+s+s1}{\PYZsq{}}\PY{l+s+s1}{Movimiento ventricular}\PY{l+s+s1}{\PYZsq{}}\PY{p}{]}\PY{p}{,} \PY{n}{ax}\PY{o}{=}\PY{n}{axes}\PY{p}{[}\PY{l+m+mi}{1}\PY{p}{]}\PY{p}{[}\PY{l+m+mi}{1}\PY{p}{]}\PY{p}{)}
        \PY{n}{fs\PYZus{}hist} \PY{o}{=} \PY{n}{sns}\PY{o}{.}\PY{n}{distplot}\PY{p}{(}\PY{n}{df\PYZus{}deceased}\PY{p}{[}\PY{l+s+s1}{\PYZsq{}}\PY{l+s+s1}{Acortamiento fraccionario}\PY{l+s+s1}{\PYZsq{}}\PY{p}{]}\PY{p}{,} \PY{n}{ax}\PY{o}{=}\PY{n}{axes}\PY{p}{[}\PY{l+m+mi}{1}\PY{p}{]}\PY{p}{[}\PY{l+m+mi}{2}\PY{p}{]}\PY{p}{)}
        
        \PY{n}{age\PYZus{}hist}\PY{o}{.}\PY{n}{set\PYZus{}title}\PY{p}{(}\PY{l+s+s1}{\PYZsq{}}\PY{l+s+s1}{Edades de no sobrevivientes al año}\PY{l+s+s1}{\PYZsq{}}\PY{p}{)}
        \PY{n}{wm\PYZus{}hist}\PY{o}{.}\PY{n}{set\PYZus{}title}\PY{p}{(}\PY{l+s+s1}{\PYZsq{}}\PY{l+s+s1}{Movimientos de no sobrevivientes al año}\PY{l+s+s1}{\PYZsq{}}\PY{p}{)}
        \PY{n}{fs\PYZus{}hist}\PY{o}{.}\PY{n}{set\PYZus{}title}\PY{p}{(}\PY{l+s+s1}{\PYZsq{}}\PY{l+s+s1}{A.F. de no sobrevivientes al año}\PY{l+s+s1}{\PYZsq{}}\PY{p}{)}
        \PY{n+nb}{print}\PY{p}{(}\PY{p}{)}
\end{Verbatim}

    \begin{Verbatim}[commandchars=\\\{\}]


    \end{Verbatim}

    \begin{center}
    \adjustimage{max size={0.9\linewidth}{0.9\paperheight}}{output_13_1.png}
    \end{center}
    { \hspace*{\fill} \\}
    
    A primera vista los datos parecen prometedores. Ahora realizamos una
prueba de shapiro para probar la normalidad de nuestras variables.

    \begin{Verbatim}[commandchars=\\\{\}]
{\color{incolor}In [{\color{incolor}13}]:} \PY{n}{alpha} \PY{o}{=} \PY{l+m+mf}{0.05}
         \PY{k}{for} \PY{n}{var} \PY{o+ow}{in} \PY{p}{(}\PY{l+s+s1}{\PYZsq{}}\PY{l+s+s1}{Edad}\PY{l+s+s1}{\PYZsq{}}\PY{p}{,} \PY{l+s+s1}{\PYZsq{}}\PY{l+s+s1}{Movimiento ventricular}\PY{l+s+s1}{\PYZsq{}}\PY{p}{,} \PY{l+s+s1}{\PYZsq{}}\PY{l+s+s1}{Acortamiento fraccionario}\PY{l+s+s1}{\PYZsq{}}\PY{p}{)}\PY{p}{:}
             \PY{n}{stat}\PY{p}{,} \PY{n}{p} \PY{o}{=} \PY{n}{scipy}\PY{o}{.}\PY{n}{stats}\PY{o}{.}\PY{n}{shapiro}\PY{p}{(}\PY{n}{df\PYZus{}alive}\PY{p}{[}\PY{n}{var}\PY{p}{]}\PY{p}{)}
             \PY{n+nb}{print}\PY{p}{(}\PY{l+s+s1}{\PYZsq{}}\PY{l+s+s1}{Estadistico=}\PY{l+s+si}{\PYZpc{}.3f}\PY{l+s+s1}{, p=}\PY{l+s+si}{\PYZpc{}.3f}\PY{l+s+s1}{\PYZsq{}} \PY{o}{\PYZpc{}} \PY{p}{(}\PY{n}{stat}\PY{p}{,} \PY{n}{p}\PY{p}{)}\PY{p}{)}
             \PY{k}{if} \PY{n}{p} \PY{o}{\PYZgt{}} \PY{n}{alpha}\PY{p}{:}
                 \PY{n+nb}{print}\PY{p}{(}\PY{l+s+s1}{\PYZsq{}}\PY{l+s+s1}{La muestra de }\PY{l+s+si}{\PYZob{}\PYZcb{}}\PY{l+s+s1}{ para pacientes que sobrevivieron }\PY{l+s+s1}{\PYZsq{}}
                       \PY{l+s+s1}{\PYZsq{}}\PY{l+s+s1}{parece normal}\PY{l+s+s1}{\PYZsq{}}\PY{o}{.}\PY{n}{format}\PY{p}{(}\PY{n}{var}\PY{p}{)}\PY{p}{)}
             \PY{k}{else}\PY{p}{:}
                 \PY{n+nb}{print}\PY{p}{(}\PY{l+s+s1}{\PYZsq{}}\PY{l+s+s1}{La muestra de }\PY{l+s+si}{\PYZob{}\PYZcb{}}\PY{l+s+s1}{ para pacientes que sobrevivieron }\PY{l+s+s1}{\PYZsq{}}
                       \PY{l+s+s1}{\PYZsq{}}\PY{l+s+s1}{no parece normal}\PY{l+s+s1}{\PYZsq{}}\PY{o}{.}\PY{n}{format}\PY{p}{(}\PY{n}{var}\PY{p}{)}\PY{p}{)}
                 
             \PY{n}{stat}\PY{p}{,} \PY{n}{p} \PY{o}{=} \PY{n}{scipy}\PY{o}{.}\PY{n}{stats}\PY{o}{.}\PY{n}{shapiro}\PY{p}{(}\PY{n}{df\PYZus{}deceased}\PY{p}{[}\PY{n}{var}\PY{p}{]}\PY{p}{)}
             \PY{n+nb}{print}\PY{p}{(}\PY{l+s+s1}{\PYZsq{}}\PY{l+s+s1}{Estadistico=}\PY{l+s+si}{\PYZpc{}.3f}\PY{l+s+s1}{, p=}\PY{l+s+si}{\PYZpc{}.3f}\PY{l+s+s1}{\PYZsq{}} \PY{o}{\PYZpc{}} \PY{p}{(}\PY{n}{stat}\PY{p}{,} \PY{n}{p}\PY{p}{)}\PY{p}{)}
             \PY{k}{if} \PY{n}{p} \PY{o}{\PYZgt{}} \PY{n}{alpha}\PY{p}{:}
                 \PY{n+nb}{print}\PY{p}{(}\PY{l+s+s1}{\PYZsq{}}\PY{l+s+s1}{La muestra de }\PY{l+s+si}{\PYZob{}\PYZcb{}}\PY{l+s+s1}{ para pacientes que no sobrevivieron }\PY{l+s+s1}{\PYZsq{}} \PYZbs{}
                       \PY{l+s+s1}{\PYZsq{}}\PY{l+s+s1}{parece normal}\PY{l+s+s1}{\PYZsq{}}\PY{o}{.}\PY{n}{format}\PY{p}{(}\PY{n}{var}\PY{p}{)}\PY{p}{)}
             \PY{k}{else}\PY{p}{:}
                 \PY{n+nb}{print}\PY{p}{(}\PY{l+s+s1}{\PYZsq{}}\PY{l+s+s1}{La muestra de }\PY{l+s+si}{\PYZob{}\PYZcb{}}\PY{l+s+s1}{ para pacientes que no sobrevivieron }\PY{l+s+s1}{\PYZsq{}} \PYZbs{}
                       \PY{l+s+s1}{\PYZsq{}}\PY{l+s+s1}{no parece normal}\PY{l+s+s1}{\PYZsq{}}\PY{o}{.}\PY{n}{format}\PY{p}{(}\PY{n}{var}\PY{p}{)}\PY{p}{)}
\end{Verbatim}

    \begin{Verbatim}[commandchars=\\\{\}]
Estadistico=0.993, p=0.547
La muestra de Edad para pacientes que sobrevivieron parece normal
Estadistico=0.989, p=0.168
La muestra de Edad para pacientes que no sobrevivieron parece normal
Estadistico=0.989, p=0.242
La muestra de Movimiento ventricular para pacientes que sobrevivieron parece normal
Estadistico=0.988, p=0.114
La muestra de Movimiento ventricular para pacientes que no sobrevivieron parece normal
Estadistico=0.990, p=0.264
La muestra de Acortamiento fraccionario para pacientes que sobrevivieron parece normal
Estadistico=0.990, p=0.207
La muestra de Acortamiento fraccionario para pacientes que no sobrevivieron parece normal

    \end{Verbatim}

    Podemos ver que nuestras cuatro distribuciones pasan la prueba de
shapiro, lo que indica que son normales. Ya asumiendo que las
distribuciones normales sabemos que es apropiado aplicar pruebas t sobre
las variables.

    \hypertarget{pruebas-de-hipotesis}{%
\subsection{Pruebas de hipotesis}\label{pruebas-de-hipotesis}}

Vamos a realizar pruebas de hipotesis para probar si las medias de las
distribuciones son iguales o si difieren. Realizamos tres pruebas por
pares para las tres variables que estamos estudiando.

\[
H_{0}\colon \mu_{X_1} = \mu_{X_2} \\
\]

Para \(X_1\) y \(X_2\) correspondientes a \(A_1\), \(A_2\), \(B_1\),
\(B_2\), \(C_1\), \(C_2\).

Ya que estamos asumiendo normalidad y no conocemos las varianzas de las
poblaciones, la prueba que escogí fue la prueba t de Welch. Nuestro
estadistico de prueba es

\[
t \quad = \quad {\; \overline{X}_1 - \overline{X}_2 \; \over \sqrt{ \; {s_1^2 \over N_1} \; + \; {s_2^2 \over N_2} \quad }}
\]

Usamos la función \texttt{scipy.stats.ttest\_ind}, que implementa esta
prueba.

    \begin{Verbatim}[commandchars=\\\{\}]
{\color{incolor}In [{\color{incolor}8}]:} \PY{n}{alpha} \PY{o}{=} \PY{l+m+mf}{0.05}
        \PY{n+nb}{print}\PY{p}{(}\PY{l+s+s1}{\PYZsq{}}\PY{l+s+s1}{Con nivel de significancia }\PY{l+s+si}{\PYZob{}\PYZcb{}}\PY{l+s+s1}{\PYZsq{}}\PY{o}{.}\PY{n}{format}\PY{p}{(}\PY{n}{alpha}\PY{p}{)}\PY{p}{)}
        \PY{k}{for} \PY{n}{var} \PY{o+ow}{in} \PY{p}{(}\PY{l+s+s1}{\PYZsq{}}\PY{l+s+s1}{Edad}\PY{l+s+s1}{\PYZsq{}}\PY{p}{,} \PY{l+s+s1}{\PYZsq{}}\PY{l+s+s1}{Movimiento ventricular}\PY{l+s+s1}{\PYZsq{}}\PY{p}{,} \PY{l+s+s1}{\PYZsq{}}\PY{l+s+s1}{Acortamiento fraccionario}\PY{l+s+s1}{\PYZsq{}}\PY{p}{)}\PY{p}{:}
            \PY{n+nb}{print}\PY{p}{(}\PY{l+s+s1}{\PYZsq{}}\PY{l+s+s1}{Realizando la prueba t para }\PY{l+s+si}{\PYZob{}\PYZcb{}}\PY{l+s+s1}{\PYZsq{}}\PY{o}{.}\PY{n}{format}\PY{p}{(}\PY{n}{var}\PY{p}{)}\PY{p}{)}
            \PY{n}{stat}\PY{p}{,} \PY{n}{p} \PY{o}{=} \PY{n}{scipy}\PY{o}{.}\PY{n}{stats}\PY{o}{.}\PY{n}{ttest\PYZus{}ind}\PY{p}{(}\PY{n}{df\PYZus{}alive}\PY{p}{[}\PY{n}{var}\PY{p}{]}\PY{p}{,} \PY{n}{df\PYZus{}deceased}\PY{p}{[}\PY{n}{var}\PY{p}{]}\PY{p}{,} \PY{n}{axis}\PY{o}{=}\PY{l+m+mi}{0}\PY{p}{,}
                                            \PY{n}{equal\PYZus{}var}\PY{o}{=}\PY{k+kc}{False}\PY{p}{)}
            \PY{n+nb}{print}\PY{p}{(}\PY{l+s+s1}{\PYZsq{}}\PY{l+s+s1}{Estadistico=}\PY{l+s+si}{\PYZpc{}.3f}\PY{l+s+s1}{, p=}\PY{l+s+si}{\PYZpc{}.3f}\PY{l+s+s1}{\PYZsq{}} \PY{o}{\PYZpc{}} \PY{p}{(}\PY{n}{stat}\PY{p}{,} \PY{n}{p}\PY{p}{)}\PY{p}{)}
            \PY{k}{if} \PY{n}{p} \PY{o}{\PYZlt{}} \PY{n}{alpha}\PY{p}{:}
                \PY{n+nb}{print}\PY{p}{(}\PY{l+s+s1}{\PYZsq{}}\PY{l+s+s1}{Las distribuciones de }\PY{l+s+si}{\PYZob{}\PYZcb{}}\PY{l+s+s1}{ parecen tener media distinta }\PY{l+s+s1}{\PYZsq{}} \PYZbs{}
                      \PY{l+s+s1}{\PYZsq{}}\PY{l+s+s1}{(rechaza h0)}\PY{l+s+s1}{\PYZsq{}}\PY{o}{.}\PY{n}{format}\PY{p}{(}\PY{n}{var}\PY{p}{)}\PY{p}{)}
            \PY{k}{else}\PY{p}{:}
                \PY{n+nb}{print}\PY{p}{(}\PY{l+s+s1}{\PYZsq{}}\PY{l+s+s1}{Las distribuciones de }\PY{l+s+si}{\PYZob{}\PYZcb{}}\PY{l+s+s1}{ parecen ser iguales }\PY{l+s+s1}{\PYZsq{}} \PYZbs{}
                      \PY{l+s+s1}{\PYZsq{}}\PY{l+s+s1}{(no rechaza h0)}\PY{l+s+s1}{\PYZsq{}}\PY{o}{.}\PY{n}{format}\PY{p}{(}\PY{n}{var}\PY{p}{)}\PY{p}{)}
\end{Verbatim}

    \begin{Verbatim}[commandchars=\\\{\}]
Con nivel de significancia 0.05
Realizando la prueba t para Edad
Estadistico=1.032, p=0.303
Las distribuciones de Edad parecen ser iguales (no rechaza h0)
Realizando la prueba t para Movimiento ventricular
Estadistico=2.527, p=0.012
Las distribuciones de Movimiento ventricular parecen tener media distinta (rechaza h0)
Realizando la prueba t para Acortamiento fraccionario
Estadistico=-3.510, p=0.001
Las distribuciones de Acortamiento fraccionario parecen tener media distinta (rechaza h0)

    \end{Verbatim}

    Observamos que las medias de las distribuciones de edad no difieren de
forma significativa. Podemos concluir que para nuestra muestra la edad
no es indicador de si el paciente sobrevivirá al año.

Las otras dos variables fallan la prueba. Por esto podemos concluir con
95\% de significancia que las medias de estas distribuciones difieren, y
que podemos usar estas variables como indicadores de riesgo de muerte
para pacientes que sufrieron un paro cardiaco.

    \hypertarget{predicciuxf3n-de-la-mortalidad}{%
\subsection{Predicción de la
mortalidad}\label{predicciuxf3n-de-la-mortalidad}}

Habiendo realizado nuestras pruebas de hipotesis, procedemos con las dos
variables con las que tuvimos exito: el movimiento ventricular y el
acotamiento fraccionario. Comenzamos por graficar la dispersion de
nuestros puntos dispersados con respecto a la supervivencia.

    \begin{Verbatim}[commandchars=\\\{\}]
{\color{incolor}In [{\color{incolor}9}]:} \PY{n}{f}\PY{p}{,} \PY{n}{axes} \PY{o}{=} \PY{n}{plt}\PY{o}{.}\PY{n}{subplots}\PY{p}{(}\PY{l+m+mi}{1}\PY{p}{,} \PY{l+m+mi}{3}\PY{p}{,} \PY{n}{figsize}\PY{o}{=}\PY{p}{(}\PY{l+m+mi}{16}\PY{p}{,} \PY{l+m+mi}{4}\PY{p}{)}\PY{p}{)}
        \PY{n}{ax} \PY{o}{=} \PY{n}{sns}\PY{o}{.}\PY{n}{scatterplot}\PY{p}{(}\PY{n}{x}\PY{o}{=}\PY{l+s+s2}{\PYZdq{}}\PY{l+s+s2}{Movimiento ventricular}\PY{l+s+s2}{\PYZdq{}}\PY{p}{,} \PY{n}{y}\PY{o}{=}\PY{l+s+s2}{\PYZdq{}}\PY{l+s+s2}{Vivo al año}\PY{l+s+s2}{\PYZdq{}}\PY{p}{,}
                             \PY{n}{data}\PY{o}{=}\PY{n}{df\PYZus{}vars}\PY{p}{,} \PY{n}{ax}\PY{o}{=}\PY{n}{axes}\PY{p}{[}\PY{l+m+mi}{0}\PY{p}{]}\PY{p}{)}
        \PY{n}{ax}\PY{o}{.}\PY{n}{set\PYZus{}title}\PY{p}{(}\PY{l+s+s1}{\PYZsq{}}\PY{l+s+s1}{Dispersion del movimiento}\PY{l+s+s1}{\PYZsq{}}\PY{p}{)}
        \PY{n}{ax} \PY{o}{=} \PY{n}{sns}\PY{o}{.}\PY{n}{scatterplot}\PY{p}{(}\PY{n}{x}\PY{o}{=}\PY{l+s+s2}{\PYZdq{}}\PY{l+s+s2}{Acortamiento fraccionario}\PY{l+s+s2}{\PYZdq{}}\PY{p}{,} \PY{n}{y}\PY{o}{=}\PY{l+s+s2}{\PYZdq{}}\PY{l+s+s2}{Vivo al año}\PY{l+s+s2}{\PYZdq{}}\PY{p}{,}
                             \PY{n}{data}\PY{o}{=}\PY{n}{df\PYZus{}vars}\PY{p}{,} \PY{n}{ax}\PY{o}{=}\PY{n}{axes}\PY{p}{[}\PY{l+m+mi}{1}\PY{p}{]}\PY{p}{)}
        \PY{n}{ax}\PY{o}{.}\PY{n}{set\PYZus{}title}\PY{p}{(}\PY{l+s+s1}{\PYZsq{}}\PY{l+s+s1}{Dispersion del A.F}\PY{l+s+s1}{\PYZsq{}}\PY{p}{)}
        \PY{n}{ax} \PY{o}{=} \PY{n}{sns}\PY{o}{.}\PY{n}{scatterplot}\PY{p}{(}\PY{n}{x}\PY{o}{=}\PY{l+s+s2}{\PYZdq{}}\PY{l+s+s2}{Acortamiento fraccionario}\PY{l+s+s2}{\PYZdq{}}\PY{p}{,} \PY{n}{y}\PY{o}{=}\PY{l+s+s2}{\PYZdq{}}\PY{l+s+s2}{Movimiento ventricular}\PY{l+s+s2}{\PYZdq{}}\PY{p}{,}
                             \PY{n}{hue}\PY{o}{=}\PY{l+s+s2}{\PYZdq{}}\PY{l+s+s2}{Vivo al año}\PY{l+s+s2}{\PYZdq{}}\PY{p}{,} \PY{n}{data}\PY{o}{=}\PY{n}{df\PYZus{}vars}\PY{p}{,} \PY{n}{ax}\PY{o}{=}\PY{n}{axes}\PY{p}{[}\PY{l+m+mi}{2}\PY{p}{]}\PY{p}{)}
        \PY{n}{ax}\PY{o}{.}\PY{n}{set\PYZus{}title}\PY{p}{(}\PY{l+s+s1}{\PYZsq{}}\PY{l+s+s1}{Dispersion de los datos}\PY{l+s+s1}{\PYZsq{}}\PY{p}{)}
        \PY{n+nb}{print}\PY{p}{(}\PY{p}{)}
\end{Verbatim}

    \begin{Verbatim}[commandchars=\\\{\}]


    \end{Verbatim}

    \begin{center}
    \adjustimage{max size={0.9\linewidth}{0.9\paperheight}}{output_21_1.png}
    \end{center}
    { \hspace*{\fill} \\}
    
    La primera observación que podemos hacer es que aunque los puntos en
ciertas partes pueden parecen ser bastante homogeneos, hay secciones
donde se notan mas puntos de un color o del otro.

Para intentar clasificar los puntos, usaremos un modelo lineal de
regresión logistica simple. El paquete SKLearn implementa este modelo
para nuestro uso.

Separaremos los datos en conjuntos de entrenamiento y prueba para poder
comprobar nuestro trabajo.

    \begin{Verbatim}[commandchars=\\\{\}]
{\color{incolor}In [{\color{incolor}10}]:} \PY{k+kn}{from} \PY{n+nn}{sklearn}\PY{n+nn}{.}\PY{n+nn}{model\PYZus{}selection} \PY{k}{import} \PY{n}{train\PYZus{}test\PYZus{}split}
         
         \PY{n}{X\PYZus{}train}\PY{p}{,} \PY{n}{X\PYZus{}test}\PY{p}{,} \PY{n}{y\PYZus{}train}\PY{p}{,} \PY{n}{y\PYZus{}test} \PY{o}{=} \PY{n}{train\PYZus{}test\PYZus{}split}\PY{p}{(}
             \PY{n}{df\PYZus{}vars}\PY{o}{.}\PY{n}{iloc}\PY{p}{[}\PY{p}{:}\PY{p}{,} \PY{l+m+mi}{1}\PY{p}{:}\PY{o}{\PYZhy{}}\PY{l+m+mi}{1}\PY{p}{]}\PY{p}{,} \PY{n}{df\PYZus{}vars}\PY{o}{.}\PY{n}{iloc}\PY{p}{[}\PY{p}{:}\PY{p}{,} \PY{o}{\PYZhy{}}\PY{l+m+mi}{1}\PY{p}{]}\PY{p}{,} \PY{n}{test\PYZus{}size}\PY{o}{=}\PY{l+m+mf}{0.2}\PY{p}{)}
         
         \PY{n+nb}{print}\PY{p}{(}\PY{l+s+s1}{\PYZsq{}}\PY{l+s+s1}{Tamaño del conjunto de entrenamiento: }\PY{l+s+si}{\PYZob{}\PYZcb{}}\PY{l+s+s1}{\PYZsq{}}\PY{o}{.}\PY{n}{format}\PY{p}{(}\PY{n}{X\PYZus{}train}\PY{o}{.}\PY{n}{shape}\PY{p}{[}\PY{l+m+mi}{0}\PY{p}{]}\PY{p}{)}\PY{p}{)}
         \PY{n+nb}{print}\PY{p}{(}\PY{l+s+s1}{\PYZsq{}}\PY{l+s+s1}{Tamaño del conjunto de prueba: }\PY{l+s+si}{\PYZob{}\PYZcb{}}\PY{l+s+s1}{\PYZsq{}}\PY{o}{.}\PY{n}{format}\PY{p}{(}\PY{n}{X\PYZus{}test}\PY{o}{.}\PY{n}{shape}\PY{p}{[}\PY{l+m+mi}{0}\PY{p}{]}\PY{p}{)}\PY{p}{)}
\end{Verbatim}

    \begin{Verbatim}[commandchars=\\\{\}]
Tamaño del conjunto de entrenamiento: 280
Tamaño del conjunto de prueba: 70

    \end{Verbatim}

    Ajustamos los modelos. Ya que tenemos esto, revisamos la precision de
nuestro clasificador en el conjunto de prueba. Imprimimos ademas la
mátriz de confusión para ver el comportamiento del clasificador lineal.
Comenzamos por

    \begin{Verbatim}[commandchars=\\\{\}]
{\color{incolor}In [{\color{incolor}9}]:} \PY{n}{f}\PY{p}{,} \PY{n}{axes} \PY{o}{=} \PY{n}{plt}\PY{o}{.}\PY{n}{subplots}\PY{p}{(}\PY{l+m+mi}{1}\PY{p}{,} \PY{l+m+mi}{3}\PY{p}{,} \PY{n}{figsize}\PY{o}{=}\PY{p}{(}\PY{l+m+mi}{16}\PY{p}{,} \PY{l+m+mi}{4}\PY{p}{)}\PY{p}{)}
        \PY{n}{ax} \PY{o}{=} \PY{n}{sns}\PY{o}{.}\PY{n}{scatterplot}\PY{p}{(}\PY{n}{x}\PY{o}{=}\PY{l+s+s2}{\PYZdq{}}\PY{l+s+s2}{Movimiento ventricular}\PY{l+s+s2}{\PYZdq{}}\PY{p}{,} \PY{n}{y}\PY{o}{=}\PY{l+s+s2}{\PYZdq{}}\PY{l+s+s2}{Vivo al año}\PY{l+s+s2}{\PYZdq{}}\PY{p}{,}
                             \PY{n}{data}\PY{o}{=}\PY{n}{df\PYZus{}vars}\PY{p}{,} \PY{n}{ax}\PY{o}{=}\PY{n}{axes}\PY{p}{[}\PY{l+m+mi}{0}\PY{p}{]}\PY{p}{)}
        \PY{n}{ax}\PY{o}{.}\PY{n}{set\PYZus{}title}\PY{p}{(}\PY{l+s+s1}{\PYZsq{}}\PY{l+s+s1}{Dispersion del movimiento}\PY{l+s+s1}{\PYZsq{}}\PY{p}{)}
        \PY{n}{ax} \PY{o}{=} \PY{n}{sns}\PY{o}{.}\PY{n}{scatterplot}\PY{p}{(}\PY{n}{x}\PY{o}{=}\PY{l+s+s2}{\PYZdq{}}\PY{l+s+s2}{Acortamiento fraccionario}\PY{l+s+s2}{\PYZdq{}}\PY{p}{,} \PY{n}{y}\PY{o}{=}\PY{l+s+s2}{\PYZdq{}}\PY{l+s+s2}{Vivo al año}\PY{l+s+s2}{\PYZdq{}}\PY{p}{,}
                             \PY{n}{data}\PY{o}{=}\PY{n}{df\PYZus{}vars}\PY{p}{,} \PY{n}{ax}\PY{o}{=}\PY{n}{axes}\PY{p}{[}\PY{l+m+mi}{1}\PY{p}{]}\PY{p}{)}
        \PY{n}{ax}\PY{o}{.}\PY{n}{set\PYZus{}title}\PY{p}{(}\PY{l+s+s1}{\PYZsq{}}\PY{l+s+s1}{Dispersion del A.F}\PY{l+s+s1}{\PYZsq{}}\PY{p}{)}
        \PY{n}{ax} \PY{o}{=} \PY{n}{sns}\PY{o}{.}\PY{n}{scatterplot}\PY{p}{(}\PY{n}{x}\PY{o}{=}\PY{l+s+s2}{\PYZdq{}}\PY{l+s+s2}{Acortamiento fraccionario}\PY{l+s+s2}{\PYZdq{}}\PY{p}{,} \PY{n}{y}\PY{o}{=}\PY{l+s+s2}{\PYZdq{}}\PY{l+s+s2}{Movimiento ventricular}\PY{l+s+s2}{\PYZdq{}}\PY{p}{,}
                             \PY{n}{hue}\PY{o}{=}\PY{l+s+s2}{\PYZdq{}}\PY{l+s+s2}{Vivo al año}\PY{l+s+s2}{\PYZdq{}}\PY{p}{,} \PY{n}{data}\PY{o}{=}\PY{n}{df\PYZus{}vars}\PY{p}{,} \PY{n}{ax}\PY{o}{=}\PY{n}{axes}\PY{p}{[}\PY{l+m+mi}{2}\PY{p}{]}\PY{p}{)}
        \PY{n}{ax}\PY{o}{.}\PY{n}{set\PYZus{}title}\PY{p}{(}\PY{l+s+s1}{\PYZsq{}}\PY{l+s+s1}{Dispersion de los datos}\PY{l+s+s1}{\PYZsq{}}\PY{p}{)}
        \PY{n+nb}{print}\PY{p}{(}\PY{p}{)}
\end{Verbatim}

    \begin{tabular}{lrr}
\toprule
{} &  No &  Si \\
\midrule
No &  29 &   5 \\
Si &  31 &   5 \\
\bottomrule
\end{tabular}


    
    \begin{Verbatim}[commandchars=\\\{\}]
              precision    recall  f1-score   support

          no       0.48      0.85      0.62        34
          si       0.50      0.14      0.22        36

   micro avg       0.49      0.49      0.49        70
   macro avg       0.49      0.50      0.42        70
weighted avg       0.49      0.49      0.41        70


    \end{Verbatim}

    Este conjunto de datos tiene la caracteristica de que tiende ser mas
facil predecir casos negativos que positivos correctamente. Es decir que
el modelo se ajustó de forma que es detecta la mayoria de los casos
negativos correctamente, pero tiene dificultades para detectar casos
positivos correctamente. Esto puede ser debido a la naturaleza de los
datos o simplemente porque la muestra es muy pequeña.

Esto se refleja en el clasificador resultante. Aunque las precision en
general no es mucho mejor que la de elegir aleatoreamente, si el
clasificador predice que un valor es falso, el clasificador detectará
esto un 85\% de las veces. Este factor de recuerdo posiblemente podria
ser mejor con una cantidad mayor de datos. Un factor de recuerdo mayor
haria que el echocardiograma pudiera ser un metodo efectivo para
descartar pacientes que probablemente no estén en riesgo. Con el
conjunto de datos actual, sin embargo, el clasificador como existe
actualmente no es suficiente para tomar decisiones concretas con niveles
de confianza significativos.

    \hypertarget{conclusiuxf3n}{%
\section{Conclusión}\label{conclusiuxf3n}}

A partir de pruebas de hipotesis pudimos observar que tanto el
movimiento de la pared vascular como el achicamiento fraccional son
factores que pueden indicar un riesgo de mortalidad en el periodo de un
año posterior a un paro cardiaco. Similarmente vimos que la edad no es
un factor de gran ayuda al momento de predecir este riesgo.

Aunque efectivamente encontramos que estas dos variables se
correlacionan con la mortalidad, no son suficientes para ajustar un
modelo que pueda predecir satisfactoriamente la mortalidad a un nivel
que pueda ser especialmente util. Sin embargo terminamos con la
conclusión de que estas medidas obtenidas del echocardiograma pueden ser
suficientes para indicarle a un médico que un paciente necesitará
cuidado especial.


    % Add a bibliography block to the postdoc
    
    
    
    \end{document}
